\documentclass[aspectratio=1610]{beamer}

\usepackage[T1]{fontenc}
\usepackage[utf8]{inputenc}

\usetheme{PaloAlto}
\usecolortheme{crane}
\usepackage{times}
%\usepackage{serif}

% Persquisar este:
% \usefonttheme[onlymath]{mathserif}

% \title{}
% \author{Hélio Giroto}

% Tira os símbolos de navegaçao no rodapé dos slides pdf:
\beamertemplatenavigationsymbolsempty

\begin{document}

\section [Capa]{}
\begin{frame}

\Huge \textbf{BASH} \vspace{0.5cm}

\large
\textbf{Guia de Sintaxes e Exemplos}

(Orientados à Programação)\vspace{1.2cm}


\begin{flushright}
	\scriptsize{\textit{Autor: Hélio Giroto}}
\end{flushright}

  
\end{frame}


\section [Bash e etc...]{}
\begin{frame}
  \frametitle{Quotings}
  \framesubtitle{}

\end{frame}


\begin{frame}
  \frametitle{Comentários}
  \framesubtitle{}

\end{frame}


\begin{frame}
  \frametitle{Comandos compostos}
  \framesubtitle{}

\end{frame}


\begin{frame}
  \frametitle{Expansões}
  \framesubtitle{(De Strings e Aritméticas)}

\end{frame}


\begin{frame}
  \frametitle{Substituições}
  \framesubtitle{(De comandos e de processos)}

\end{frame}


\begin{frame}
  \frametitle{Padrões}
  \framesubtitle{}

\end{frame}


\begin{frame}
  \frametitle{Redirecionamentos}
  \framesubtitle{}

\end{frame}


\begin{frame}
  \frametitle{EOF}
  \framesubtitle{Pg 33 ?}

\end{frame}


\section [Variáveis]{}
\begin{frame}
  \frametitle{Variáveis}
  \framesubtitle{}
  
  \begin{block}{Atribuição:}
	\begin{itemize}
	\item {\textit{Nomes de variáveis:}} 
		\begin{itemize}	
		\item[-]{Se aceita alfanuméricos e underlines (\_). }
		\item[-]{Diferencia maiúsculas de minúsculas. }
		\item[-]{Não pode começar com número.}
		\end{itemize}
	\item {\textit{Exemplos:}} \\
  		var=Linux \\
  		Var="FreeBSD" \\
  		var\_2="Bom dia para todos!" \\
  		texto="João\ \ \ \ \ \ 350,00"\\
  		valor=38 \\
	\end{itemize}
  \end{block}
  
  
  \begin{block}{Acesso:}
	  \begin{itemize}
	  	\item{\textit{É chamada com o símbolo \$dolar - Ex:}}\\
	  	echo \$var \\
	  	echo "\$texto" - *\scriptsize {(Se coloca áspas para imprimir os espaços [Ver acima])}
	  \end{itemize}
  \end{block}

\end{frame}


\section [READ]{}
\begin{frame}
  \frametitle{Comando READ}
  \framesubtitle{}

\end{frame}


\section [Parâmetros]{}
\begin{frame}
  \frametitle{Parâmetros e o comando SHIFT}
  \framesubtitle{}

\end{frame}


\section [Arrays]{}
\begin{frame}
  \frametitle{Arrays}
  \framesubtitle{}

\begin{block}{}
	\# Declarar um array:\\
	cidades=(Brasilia Manaus "São Paulo" "Rio de Janeiro" Cuiabá)
\end{block}

\begin{block}{}
	\# Exibir os elementos do array:\\
	echo \$\{cidades[0]\}\ \ \ \ \# Saída: Brasilia.\\
	echo \$\{cidades[*]\}\ \ \ \ \# Exibe \textbf{todos} os elementos do array.\\
	echo \$\{cidades[@]\}\ \  \# mesmo efeito acima...\\
	echo \$\{#cidades[*]\}\ \ \ \ \# Mostra a \textbf{quantidade} de elementos do array\\
\end{block}

\begin{block}{}
	\# Eliminar um elemento do array:\\
	unset cidades[0]\ \ \ \# Elimina o elemento 0 do array.\\
	unset cidades \ \ \ \ \ \#\ Elimina o array completo
\end{block}

* \textit{Ver como usar um array com o \textbf{comando FOR}.}


\end{frame}


\section [IF]{}
\begin{frame}
  \frametitle{Comando IF} \vspace{-0.3cm}     % comente esse vspace e veja a diferença de como o espaço aumenta na primeira linha! 
  \framesubtitle{}
\begin{columns}
% ver tb sobre o minipage: http://www.alessandroduarte.com.br/?page_id=602  ou https://texblog.org/2007/08/01/placing-figurestables-side-by-side-minipage/
% ou mesmo columns em:  https://tex.stackexchange.com/questions/32931/multiple-columns-with-images-and-wrapped-text-in-beamer   ou   https://www.overleaf.com/learn/latex/Multiple_columns
% Posição:   https://www.overleaf.com/learn/latex/Positioning_of_Figures
    \begin{column}{0.45\textwidth}   % Se usamos 0.5 aqui ao lado, as duas colunas ficam extremamente juntas (sem separação).
       \begin{block}{}
	      if [[ "\$var" == "algo" ]]\\
	      then\\
		      \textit{\ \ \ comandos...}\\
	      fi\\
        \end{block}

        \begin{block}{}
	      if [[ \$num -eq 0 ]]\\
	      then\\
		      \ \ \ echo "Nro. é zero."\\
	      else\\
		      \ \ \ echo "Nro. diferente de zero."\\
	      fi\\
        \end{block}
        
        \begin{block}{}
	      if [[ \$opcao -gt 5 ]]\\
	      then\\
		       \ \ \ echo "Opcao MAIOR QUE 5"\\
	      fi\\
        \end{block}
    \end{column}
        
\begin{column}{0.45\textwidth}
    Segunda coluna
        \begin{block}{}
	      testar exemplos - AND e OR\\
	      mais ex.: com test... com arquivos... 	\\
	      acima: dividir em duas colunas de blocos...\\
        \end{block}
\end{column}

\end{columns}
\end{frame}


\begin{frame}
  \frametitle{Expressões Condicionais}
  \framesubtitle{}
      pg. 90-92
\end{frame}


\section [Operadores]{}
\begin{frame}
  \frametitle{Operadores }
  \framesubtitle{(De valor, aritméticos, relacionais, lógicos)}

\end{frame}


\section [CASE]{}
\begin{frame}
  \frametitle{Comando CASE}
  \framesubtitle{}

	\begin{block}{}
			
	\end{block}

\end{frame}


\section [WHILE]{}
\begin{frame}
  \frametitle{Comando WHILE}
  \framesubtitle{}

\end{frame}


\section [UNTIL]{}
\begin{frame}
  \frametitle{Comando UNTIL}
  \framesubtitle{}

\end{frame}


\section [FOR]{}
\begin{frame}
  \frametitle{Comando FOR}
  \framesubtitle{}

\end{frame}


\section [SELECT]{}
\begin{frame}
  \frametitle{Comando SELECT}
  \framesubtitle{}

\end{frame}


\section [Funções]{}
\begin{frame}
  \frametitle{Funções}
  \framesubtitle{}

\end{frame}


\section [Considerações]{}
\begin{frame}
  \frametitle{Considerações Finais}
  \maketitle

\end{frame}


\end{document}

% VER TB:
% https://tex.stackexchange.com/questions/167714/writing-source-code-in-latex-as-text
