\documentclass[aspectratio=1610]{beamer}
\setbeamertemplate{background} {\includegraphics[width=\paperwidth,height=\paperheight,keepaspectratio]{backg.jpg}}
%https://tex.stackexchange.com/questions/78464/background-image-in-beamer-slides

\usepackage[utf8]{inputenc}

\usetheme{Goettingen}
%\usetheme{Bergen}
\usecolortheme{whale}
% \usepackage{serif}
\beamertemplatenavigationsymbolsempty

\title{TIPOS DE ARGUMENTOS}
\author[Passei Direito]{https://www.passeidireto.com/arquivo/1186810/principais-tipos-de-argumento}
% ========================================================================
\begin{document}
\maketitle
\tableofcontents 

% ========================================================================
\begin{frame}
  \frametitle{PRINCIPAIS TIPOS DE ARGUMENTO}
\end{frame}

% ========================================================================
\section {SENSO COMUM}
\begin{frame}
  \frametitle{ARGUMENTO DE SENSO COMUM   }
\begin{center}
É aquele que invoca princípio genérico, indiscutível, conhecido por toda a sociedade, Seu efeito suasório é reduzido e, por isso, ele deve ser utilizado como reforço a um argumento mais específico.
\end{center}

\begin{itemize}
  \item \textit{EXEMPLO}: Sabe-se que quem tem o nome incluído no SPC passa por uma situação constrangedora. Sendo assim, aquele cujo nome foi inserido indevidamente nesse cadastro tem direito à indenização por danos morais.
\end{itemize}
\end{frame}

% ========================================================================
\section {ANALOGIA}
\begin{frame}
  \frametitle{ARGUMENTO POR ANALOGIA  }
\begin{center}
  Aparece principalmente no uso das decisões jurisprudenciais e baseia-se no princípio de que a justiça deve tratar de maneira semelhante situações idênticas.
 \end{center}

\begin{itemize}
  \item \textit{EXEMPLO}: Se o dono de um estabelecimento comercial é obrigado a pagar tributos para ter o direito de vender sua mercadoria, por analogia, um camelô também deveria pagar tributos sobre a mercadoria comercializada.
\end{itemize}
\end{frame}

% ========================================================================
\section {CONTRÁRIO}
\begin{frame}
  \frametitle{ARGUMENTO A CONTRÁRIO SENSO}
\begin{center}
  É aquele que concede a uma proposição interpretação inversa. Muito utilizado no contexto jurídico, seu uso deve ser cuidadoso para que não se aproxime da falácia.
 \end{center}

\begin{itemize}
  \item \textit{EXEMPLO}:  Se o legislador especificou taxativamente os casos de incidência do tributo, a contrário senso os demais casos não estão abrangidos.
\end{itemize}
\end{frame}

% ========================================================================
\section {ABSURDO}
\begin{frame}
  \frametitle{ARGUMENTO POR ABSURDO (AB ABSURDO)    }

\begin{center}
Refuta uma asserção, mostrando-lhe a falta de cabimento ao contrariar a evidência.
\end{center}

\begin{itemize}
  \item \textit{EXEMPLO}: Como poderia a mulher ter alvejado o marido, se o laudo médico atesta que ela morreu minutos antes do esposo?
\end{itemize}
\end{frame}

% ========================================================================
\section {EXCLUSÃO}
\begin{frame}
  \frametitle{ARGUMENTO POR EXCLUSÃO (PER EXCLUSIONEM)   }
\begin{center}
Propõem-se várias hipóteses e vai-se eliminando uma a uma.
\end{center}

\begin{itemize}
  \item \textit{EXEMPLO}: Poder-se-ia afirmar que o réu não é capaz de controlar os seus atos, mas soube premeditar o crime.
\end{itemize}
\end{frame}

% ========================================================================
\section {A POSTERIORI}
\begin{frame}
  \frametitle{ARGUMENTO A POSTERIORI   }
\begin{center}
Consiste em desenvolver um raciocínio, admitido como mais claro, de expor as consequências de um fato, permitindo voltar às causas, eventualmente menos conhecidas do caso em tela.
\end{center}

\begin{itemize}
  \item \textit{EXEMPLO}: Sabe-se que o pai desenvolveu comportamento possessivo em relação aos filhos e os culpava pelos acontecimentos e infortúnios, consequência imediata da esquizofrenia paranoide que o acometia.
\end{itemize}
\end{frame}

% ========================================================================
\section {CAUSA-EFEITO}
\begin{frame}
  \frametitle{ARGUMENTO DE CAUSA E EFEITO }
\begin{center}
  Relaciona conceitos de causalidade e efeito com o objetivo de evidenciar as consequências imediatas de determinado ato (retirado das provas) praticado pelas partes.
 \end{center}

\begin{itemize}
  \item \textit{EXEMPLO}: Já que a vítima não possui automóvel e trabalha até tarde como vendedora em um shopping há 1 hora e meia de casa, não poderia ela deixar de passar por tal lugar que, apesar de ermo, é caminho obrigatório para sua casa.
\end{itemize}
\end{frame}

% ========================================================================
\section {PROVA}
\begin{frame}
  \frametitle{ARGUMENTO DE PROVA   }
\begin{center}
É aquele que explora a prova testemunhal, e é tão mais persuasivo quanto maior for a credibilidade e isenção de interesses do testemunho prestado. 
\end{center}

\textit{OBSERVAÇÃO}: A prova técnica, quando aceita como verdadeira, transforma-se em prova concreta, indiscutível, em geral, ela não consegue resolver todos...
\begin{itemize}
  \item \textit{EXEMPLO}: ... ninguém viu o acusado pulando o muro da sua casa, tampouco ouviu-se o grito da menina, comprovando a improcedência da acusação do réu.
\end{itemize}
\end{frame}

% ========================================================================
\section {AUTORIDADE}
\begin{frame}
  \frametitle{ARGUMENTO DE AUTORIDADE (EX/AB AUCTORITATE)   }
\begin{center}
É aquele que invoca lição de pessoa conhecida e reconhecida em determinada disciplina para avaliar um posicionamento defendido na peça jurídica. Apesar de muito persuasivo, o argumento de autoridade deve ser exposto de maneira a que se comprove seu percurso lógico e não vale apenas porque é proveniente de uma pessoa conhecida.
\end{center}

OBSERVAÇÃO: Não faça de sua argumentação um amontoado de citações. Use-as apropriadamente.

\begin{itemize}
  \item \textit{EXEMPLO}: Como esclareceu o douto Procurador, Paulo Cezar Pinheiro Carneiro,  é preciso deixar claro que, na separação consensual, a sociedade conjugal é dissolvida.
\end{itemize}
\end{frame}

% ========================================================================
\section {FORTIORI}
\begin{frame}
  \frametitle{ARGUMENTO A FORTIORI   }
\begin{center}
Fundamenta-se na asserção de que se a lei proíbe ou permite determinada conduta, com maior razão proíbe ou permite uma conduta maior ou menor, respectivamente.
\end{center}

\begin{itemize}
  \item \textit{EXEMPLO}: Se a negligência deve ser punida, tanto mais o ato premeditado.
\end{itemize}
\end{frame}

% ========================================================================
\section {FUGA}
\begin{frame}
  \frametitle{ARGUMENTO DE FUGA   }
\begin{center}
É aquele que se desvia das questões principais que devem ser defendidas para buscar sensibilização por meio de temas mais subjetivos. Seu uso deve ser comedido, pois se tem forte tendência a confundir o relatório.
\end{center}

\begin{itemize}
  \item \textit{EXEMPLO}: O advogado habilidoso, que não tem como negar o crime do réu, enfatiza que ele é bom filho, bom marido, trabalhador, etc.
\end{itemize}
\end{frame}

% ========================================================================
\section {COERÊNCIA}
\begin{frame}
  \frametitle{ARGUMENTO DE COERÊNCIA (A COHERENTIA)   }
\begin{center}
É aquele que usa da asserção de que dois preceitos normativos não podem regular a mesma situação física.
\end{center}

\begin{itemize}
  \item \textit{EXEMPLO}: Não é coerente qualificar um único homicídio de doloso e culposo ao mesmo tempo.
\end{itemize}
\end{frame}

% ========================================================================
\section {AD HOMINEM}
\begin{frame}
  \frametitle{ARGUMENTO CONTRA O HOMEM (AD HOMINEM)   }
\begin{center}
Ocorre quando consideramos errada uma conclusão porque parte de alguém por nós depreciado. Ao refutar a verdade, atacamos o homem que fez a afirmação.
\end{center}

\begin{itemize}
  \item \textit{EXEMPLO}: Este técnico não tem competência para emitir um parecer sobre tal assunto. Afinal, foi o primeiro parecer que ele realizou.
\end{itemize}
\end{frame}

% ========================================================================
\begin{frame}
  \frametitle{}
\begin{center}
\textbf{https://www.passeidireto.com/arquivo/1186810/principais-tipos-de-argumento}
\end{center}
\end{frame}
\end{document}

% =========================================================================
% By: Helio Giroto
